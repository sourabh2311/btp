\chapter{Phase II : Constructing Intermediate Representation Tree}

Now that we have constructed our Abstract Syntax Tree, we will now have to convert it to an Intermediate Representation to solve $m \times n$ problem and also code generation would be easy on this language as it is more close to many machine languages.

\section{Type Interface}

\tfbox{\href{\git types.sml}{types.sml}}

Our next step is to do type checking of our input AST. For this purpose we need to define auxiliary types. Thus, we have \texttt{types.sml} representing the same. Thus now we can easily map a "symbol" to its corresponding type.

\section{Types/Variables Environment}

\tfbox{\href{\git env.sml}{env.sml} \\ \href{\git env.sig}{env.sig} \\ \href{\git temp.sml}{temp.sml} \\ \href{\git temp.sig}{temp.sig}}

Now for type checking, we need some environment under which we have to check compatibility of types as for instance, environment can be different for different functions as within \texttt{let} block of each function there can be new declarations etc. By default we have some standard types and functions already present in our environment and are thus listed in \texttt{env.sml}. Now each of these functions will correspond to some label in our MIPS code, for which we have \texttt{temp.sml}. Note that \texttt{temp.sml} is also used to represent our infinite register in Intermediate Representation.

\section{Finding Escaped Variables}

\tfbox{\href{\git findescape.sml}{findescape.sml}}

It would be important for us to determine which variable cannot go to a register, i.e., which variable escapes. Idea of finding escapes is straight forward: Just see if this thing was defined in outer scope, note that scope extends only when we are calling a function. Escaping is calculated in \texttt{findescape.sml}

\section{Intermediate Tree Representation}

\tfbox{\href{\git tree.sml}{tree.sml} \\ \href{\git tree.sig}{tree.sig}}

The files \texttt{tree.sml, tree.sig} consists of datatype representing our Intermediate Tree. Description of which is presented as below.

\begin{minted}{sml}
  datatype stm = SEQ of stm * stm (* The statement s2 followed by s2. *)
  | LABEL of label (* Define the constant value of name n to be the current machine code address. This is like a label definition in assembly language. *)
  | JUMP of exp * label list (* Transfer control (jump) to address exp. The destination exp may be a literal label, as in NAME(lab), or it may be an address calculated by any other kind of expression. For example, a C-language switch (i) statement may be implemented by doing arithmetic on i. The list of labels labs specifies all the possible locations that the expression exp can evaluate to; this is necessary for dataflow analysis later. The common case of jumping to a known label "l" is written as jump(name l, [l]). *)
  | CJUMP of relop * exp * exp * label * label (* CJUMP(o, e1, e2, t, f): Evaluate e1, e2 in that order, yielding values a, b. Then compare a, b using the relational operator o. If the result is true, jump to t otherwise jump to f. The relational operators *)
  | MOVE of exp * exp 
  (*
     MOVE(TEMP t, e): Evaluate e and move it into temporary t. 
     MOVE(MEM(e1), e2) Evaluate e1 yielding address a. Then evaluate e2 and store the result into wordSize bytes of memory starting at a.
  *)
  | EXP of exp (* Evaluate exp and discard the result. *)

and exp = BINOP of binop * exp * exp (* The application of binary operators to operands exp1, exp2. *)
  | MEM of exp (* The contents of wordSize bytes of memory starting at address exp (where wordSize is defined in the Frame module). Note that when MEM is used as the left child of a move, it means "store," but anywhere else it means "fetch." *)
  | TEMP of Temp.temp (* Temporary t. A temporary in the abstract machine is similar to a register in a real machine. However, the abstract machine has an infinite number of temporaries. *)
  | ESEQ of stm * exp (* The statement s is evaluated for side effects, then e is evaluated for result *)
  | NAME of label (* The value NAME(n) may be the target of jumps, calls, etc. *)
  | CONST of int (* The integer constant int. *)
  | CALL of exp * exp list (* A procedure call: the application of function exp1 to argument list exp2 list. The subexpression exp1 is evaluated before the arguments which are evaluated left to right. *)

and binop = PLUS | MINUS | MUL | DIV | AND | OR | LSHIFT | RSHIFT | ARSHIFT | XOR

and relop = EQ | NE | LT | GT | LE | GE | ULT | ULE | UGT | UGE
\end{minted}

\section{Understanding Functions Calls in Tiger}

\begin{tcolorbox}
  Note that some of this material will become clear after reading Phase III.
\end{tcolorbox}

Below is explained in chronological sequence of what happens and for
what reason when function call is executed in this compiler.

\begin{enumerate}
\def\labelenumi{\arabic{enumi}.}
\item
  When a function is called. The current frame is extended to include
  outgoing parameters (at the offset already determined by the callee)
  in case some of them escape; rest of the arguments are put in argument
  registers.

  \begin{enumerate}
  \def\labelenumii{\arabic{enumii}.}
  \item
    This step is done by our code generator.
  \item
    After moving escaped arguments to their pre-determined location and
    remaining arguments to argument registers, it will then emit
    \mintinline[breaklines=true, tabsize=2, breaksymbolleft=]{text}{jal}
    instruction which will have
    \mintinline[breaklines=true, tabsize=2, breaksymbolleft=]{text}{src = argTemps}
    (\mintinline[breaklines=true, tabsize=2, breaksymbolleft=]{text}{argTemps}
    are chosen argument registers, this is done as these argument
    registers are used in this time) and
    \mintinline[breaklines=true, tabsize=2, breaksymbolleft=]{text}{dst = F.getFirstL F.callersaves};
    as we know that caller is supposed to save some registers if it was
    using them and thus setting
    \mintinline[breaklines=true, tabsize=2, breaksymbolleft=]{text}{dst = F.getFirstL F.callersaves}
    would enable the garbage collection to know that if function being
    called \textbf{uses} these registers then it would clearly
    interfere.
  \item
    Let
    \mintinline[breaklines=true, tabsize=2, breaksymbolleft=]{text}{fp},
    \mintinline[breaklines=true, tabsize=2, breaksymbolleft=]{text}{sp}
    denote the current frame and stack pointer.
  \item
    To extend the current frame, we must subtract it by the amount
    \mintinline[breaklines=true, tabsize=2, breaksymbolleft=]{text}{escaped-arguments * word-size}.
    But since we should as well store the old frame pointer (as we will
    update it with the current stack pointer); we must subtract
    \mintinline[breaklines=true, tabsize=2, breaksymbolleft=]{text}{sp}
    by
    \mintinline[breaklines=true, tabsize=2, breaksymbolleft=]{text}{(escaped-arguments + 1) * word-size}.
  \item
    Now old value of fp is saved in 0th location of this
    \mintinline[breaklines=true, tabsize=2, breaksymbolleft=]{text}{sp},
    and other arguments are saved respectively at the offsets already
    determined by the callee. (Callee predetermined these offsets
    considering the fact that we will save old
    \mintinline[breaklines=true, tabsize=2, breaksymbolleft=]{text}{fp}
    at 0th word)
  \item
    Now as we are moving to the frame of other function
    \mintinline[breaklines=true, tabsize=2, breaksymbolleft=]{text}{fp}
    is updated to
    \mintinline[breaklines=true, tabsize=2, breaksymbolleft=]{text}{sp},
    and
    \mintinline[breaklines=true, tabsize=2, breaksymbolleft=]{text}{sp}
    value will be deducted by the amount needed by the callee stack.
  \item
    Thus local variables allocated will be referred with negative offset
    wrt to
    \mintinline[breaklines=true, tabsize=2, breaksymbolleft=]{text}{fp}
    and escaped argument parameters will be referred with non negative
    offset wrt
    \mintinline[breaklines=true, tabsize=2, breaksymbolleft=]{text}{fp}.
  \item
    Thus it is evident that our code generator would need to know the
    access list of the callee which wasn't there in the design mentioned
    in the book, so I added it in
    \mintinline[breaklines=true, tabsize=2, breaksymbolleft=]{text}{tree.sml}
    but to avoid cyclic dependency between
    \mintinline[breaklines=true, tabsize=2, breaksymbolleft=]{text}{tree.sml}
    and
    \mintinline[breaklines=true, tabsize=2, breaksymbolleft=]{text}{riscframe.sml},
    I had to redefine
    \mintinline[breaklines=true, tabsize=2, breaksymbolleft=]{text}{access}
    and had to create
    \mintinline[breaklines=true, tabsize=2, breaksymbolleft=]{text}{accessConv.sml}
    to facilitate conversion between access of
    \mintinline[breaklines=true, tabsize=2, breaksymbolleft=]{text}{tree}
    and
    \mintinline[breaklines=true, tabsize=2, breaksymbolleft=]{text}{riscframe}.
  \item
    Also since code generator must save escaped arguments with respect
    to the new frame pointer which is not equal to current frame pointer
    as this whole thing is updated in
    \mintinline[breaklines=true, tabsize=2, breaksymbolleft=]{text}{procEntryExit3},
    but since new frame pointer = current stack pointer - (escape-count
    + 1) * word-size, I created new function
    \mintinline[breaklines=true, tabsize=2, breaksymbolleft=]{text}{callexp}
    in
    \mintinline[breaklines=true, tabsize=2, breaksymbolleft=]{text}{riscframe}
    to do this arithmetic.
  \end{enumerate}
\item
  Now inside this called function, we must create new temporary for each
  passed argument \textbf{in argument register} and move that register's
  value to this new temporary. This might seem unnecessary but consider
  \mintinline[breaklines=true, tabsize=2, breaksymbolleft=]{text}{function m(x : int, y : int) = (h(y, y); h(x, x))}.
  If \mintinline[breaklines=true, tabsize=2, breaksymbolleft=]{text}{x}
  stays in ``parameter register 1'' throughout
  \mintinline[breaklines=true, tabsize=2, breaksymbolleft=]{text}{m},
  and \mintinline[breaklines=true, tabsize=2, breaksymbolleft=]{text}{y}
  is passed to
  \mintinline[breaklines=true, tabsize=2, breaksymbolleft=]{text}{h} in
  parameter register 1, then there is a problem. The register allocator
  will eventually choose which machine register should hold the
  temporary. If there is no interference of the type shown in function
  \mintinline[breaklines=true, tabsize=2, breaksymbolleft=]{text}{m},
  then (on the RISC) the allocator will take care to choose register the
  same register as temporary to hold that register (not implemented as of now). Then the move
  instructions will be unnecessary and will be deleted at that time.

  \begin{enumerate}
  \def\labelenumii{\arabic{enumii}.}
  \item
    This is done in
    \mintinline[breaklines=true, tabsize=2, breaksymbolleft=]{text}{newFrame}
    of
    \mintinline[breaklines=true, tabsize=2, breaksymbolleft=]{text}{riscframe},
    named as
    \mintinline[breaklines=true, tabsize=2, breaksymbolleft=]{text}{shiftInstr}
    (shift instructions).
  \end{enumerate}
\item
  Called function must save callee-save registers include
  \mintinline[breaklines=true, tabsize=2, breaksymbolleft=]{text}{ra},
  this is done in
  \mintinline[breaklines=true, tabsize=2, breaksymbolleft=]{text}{procEntryExit1},
  similarly restoring them in the end of function body is as well done
  here.
\end{enumerate}

\section{RISC Frame}

\tfbox{\href{\git riscframe.sml}{riscframe.sml}}

File \texttt{riscframe.sml} contains an interface for RISC V, which finds some use in this phase, but is mainly there for code generation phase. Its applications and purpose will become clear by seeing the written code. Before proceeding please read and understand about \textbf{static links} given at page 132-134 of the \textit{Tiger Book}. Basic gist is that in languages that allow nested function declarations, the inner functions may use variables declared in outer functions. One way to accomplish this is that whenever a function $f$ is called, it can be passed a pointer to the frame of the function statically enclosing $f$; this pointer is the static link. 

\section{Translating to IR}

\tfbox{\ghref{translate.sml}}

Done by \texttt{translate.sml} and is used by \texttt{semant.sml} to generate IR tree code. 

Its not reasonable to translate expression of our AST to an expression of IR as this is true only for certain kinds of expressions, the ones that  compute a value. Expressions that return no value (such as some procedure calls, or while expressions in the Tiger language) are more naturally represented by \texttt{Tree.stm}. And expressions with Boolean values, such as $a > b$, might best be represented as a conditional jump - a combination of \texttt{Tree.stm} and a pair of destinations represented by \texttt{Temp.labels}. Therefore, we will make a datatype exp in the Translate module to model these three kinds of expressions:

\begin{minted}{sml}
datatype exp =  Ex of Tr.exp
							| Nx of Tr.stm
              | Cx of Temp.label * Temp.label -> Tr.stm
\end{minted}

\begin{itemize}

\item \textbf{Ex} stands for an "expression," represented as a \texttt{Tree.exp}.
\item \textbf{Nx} stands for "no result," represented as a Tree statement.
\item \textbf{Cx} stands for "conditional," represented as a function from label-pair to  statement. If you pass it a true-destination and a false-destination, it will make a statement that evaluates some conditionals and then jumps to one of the  destinations (the statement will never "fall through"). For example, the Tiger expression 
  \begin{minted}{sml}
  a > b | c < d
  \end{minted} 
  might translate to the  conditional:
  \begin{minted}{sml}
  Cx(fn (t,f) => SEQ(CJUMP(GT, a, b, t, z),
  SEQ(LABEL z, CJUMP (LT, c, d, t, f))))
  \end{minted}
  for some new label \texttt{z}.
\end{itemize}

Sometimes we will have an expression of one kind and we will need to convert it to an equivalent expression of another kind. For example, the Tiger statement
\begin{minted}{sml}
flag := (a>b | c<d)
\end{minted}
. This is achieved by \texttt{unEx}. Similarly we have \texttt{unNx} and \texttt{unCx}.

Similarly each function call defines a level which is encapsulated as 
\begin{minted}{sml}
datatype level =  Top
                | Lev of {parent: level, frame: F.frame} * unit ref
\end{minted} 

\texttt{unit ref} is used to easily check for equality.

All of the functions of \texttt{translate.sml} are straight forward so please see and understand them.

\section{Semantic Analysis}

\tfbox{\ghref{semant.sml}}

After we receive our AST, we run semant's (in \texttt{semant.sml}) function \texttt{transProg} on it which will do type checking of our AST and simultaneously convert it into our IR. Both this and \texttt{translate.sml} are big so I have not put their code here, however please see linked code which is properly documented and thus is easy to understand.