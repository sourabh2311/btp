\chapter{Introduction}
\pagenumbering{arabic}\hspace{3mm}

During $6^{th}$ Semester, I wrote a compiler to compile Tiger to MIPS. Now is my attempt to build upon this compiler, to improve its functionality, efficiency and fix various issues/bugs. Also now instead of MIPS, I'll be compiling to RISC V.

\section{Issues to resolve}


\begin{itemize}
  \item Currently the number of arguments that can be given to a function is exactly the number of available argument registers. This is to be fixed to support any number of function arguments. 
  \item Currently implementation of register allocation isn't completely as what is given in the book. My register allocator lacks coalescing and is therefore incomplete. This algorithm was written in hurry last semester and is to be implemented with better heuristics.
  \item Basic blocks has to optimized with optimizations like constant propagation, constant folding, strength reduction etc.
  \item Error messages has to improved in semantic phase. Possible improvement can be to give suggestion for basic typos. 
  \item To implement additional arithmetic operations like left/right shift.
  \item String comparison has to be made as simple as "str1 $>$ str2", etc.,  instead of calling the string comparison functions to determine it.
  \item To implement ability to include pre-written code (header) files.
  \item To implement garbage collection.
  \item Add support for compile time (initial) arguments.
  \item Automated testing using Travis.
  \item And much more is possible, its like in a product.
\end{itemize}

\section{Work done this semester}

For this semester, my main focus is on translating everything from MIPS to RISC V (along with removing redundancy and refactoring of code) and if there is time then I'll work on the issues mentioned above. Translating to RISC V is taking time as I am doing three things simultaneously, viz., Recollecting what I did previously, simultaneously refactoring and converting those files to RISC V and writing a good documentation for it so that any new comer can easily grasp it with minimum prerequisites. 

\section{What has been achieved this Semester}

\begin{enumerate}
  \item Successfully translated compiler functionality from MIPS to RISC V.
    \begin{enumerate} 
      \item Now the compiled code is generated based on RISC V machine and corresponding ISA.
      \item During this process, lots of code refactoring is done along with improvement in time complexity of various intermediate computations. Like instead of finding an element in a list, a red black map is used, etc.  
      \item Files such as \href{https://github.com/sourabh2311/btp/blob/master/Compiler/runtime.s}{runtime.s} and \href{https://github.com/sourabh2311/btp/blob/master/Compiler/riscframe.sml}{riscframe.sml} were completely rewritten along with various modifications required at other places.
      \item Implemented improvements in lexical phase to detect more errors; errors in lexical phase are reported immediately resulting in program termination unlike in semantic analysis where a guess is made to facilitate printing all errors in the end.
    \end{enumerate}
  \item Wrote complete documentation of my compiler at \href{https://tigercompiler.ml}{tigercompiler.ml}. This is done to help me and anyone interested in this project to quickly revise the fundamentals and understand the working of this compiler.

  \item Wrote automated testing using Travis. See \href{https://travis-ci.org/sourabh2311/btp}{this}. Now I'll be able to see whether my changes don't break the existing functionalities and also it is useful in case someone sends a pull request.

  \item \textbf{Fixed} a major bug; Initially my compiler supported only fixed number of arguments (same as number of argument registers in the machine). Now this has been extended to support any number of arguments. During this process I have as well figured out how to completely remove \texttt{fp} register as it is as such obsolete.   
  \item Implemented multiplication by power of 2 optimization inside basic block thus laid foundation for other basic block optimizations like constant propagation, constant folding.
  \item Started work on giving a guess of literal in case of small typo. 
  \begin{itemize}
    \item Thinking of printing suggestions which are atmost 2 \href{https://en.wikipedia.org/wiki/Edit_distance}{distance} apart. This will bring the time complexity of standard DP approach of $O(n^2)$ to just $O(n)$. 
    \item Currently the issue is that to implement this, I would have to do lots of modification of the current code. Although I have \href{https://github.com/sourabh2311/btp/commit/8f27478a3c51b9e41bef68961a28c400d4ef29dd}{abstracted} out \textit{Not Found} error messages out with the environment, what is just left is to compare the literal with those of nearby length in the environment. 
    \item But to efficiently get those strings of nearby length there should be a map which maps lengths to the string set. (An array indexed by length may as well be used) To do this, I would have to define additional data structures and put them at appropriate place. 
    \item The main issue is that in the current design, environment just have integers mapped to environment entry. We got this integer by mapping string to counter, not storing the reverse map. This can be worked around by using \href{http://www.cs.utah.edu/~mjones/sml-nj-lib/atom.html}{Atom}.  
  \end{itemize}
  \item Started work on improving my Register Allocator. Current version is a bit simplified version of the algorithm mentioned in the text and is without coalescing.
\end{enumerate}

\section{Organization of The Report}

The code written for this compiler is enormous and this report gives documentation on each of these files in an understandable order. But before proceeding it is important to read about Tiger language from \href{https://www.cs.princeton.edu/~appel/modern/ml/}{this book} (read Tiger Reference Manual in Appendix). 

