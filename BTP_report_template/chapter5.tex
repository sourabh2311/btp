\chapter{Phase III : Generating Assembly Code}

This final phase generates the machine assembly code and thus completing the purpose of compiler. 
It involves the discussion of various concepts such as Canonisation, Instruction Selection, Liveness Analysis and Register Allocation.

Each of these is explained in an understandable order below.

\section{Canonisation}

\tfbox{\ghref{canon.sml}}

\hypertarget{abstract}{%
\subsection{Abstract}\label{abstract}}

It's useful to be able to evaluate the sub-expressions of an expression
in any order. If tree expressions did not contain
\mintinline[breaklines=true, tabsize=2, breaksymbolleft=]{text}{ESEQ}
and
\mintinline[breaklines=true, tabsize=2, breaksymbolleft=]{text}{CALL}
nodes, then the order of evaluation would not matter.

\hypertarget{why-call-nodes-are-an-issue}{%
\subsection{\texorpdfstring{Why
\mintinline[breaklines=true, tabsize=2, breaksymbolleft=]{text}{CALL}
nodes are an
issue?}{Why  nodes are an issue?}}\label{why-call-nodes-are-an-issue}}

In actual implementation,
\mintinline[breaklines=true, tabsize=2, breaksymbolleft=]{text}{CALL}
nodes will return value in the same register
(\mintinline[breaklines=true, tabsize=2, breaksymbolleft=]{text}{a0} in
case of RISC V). Thus in an expression like
\mintinline[breaklines=true, tabsize=2, breaksymbolleft=]{text}{BINOP(PLUS, CALL(...), CALL(...))};
the second call will overwrite the
\mintinline[breaklines=true, tabsize=2, breaksymbolleft=]{text}{a0}
register before the
\mintinline[breaklines=true, tabsize=2, breaksymbolleft=]{text}{PLUS}
can be executed.

Remedy is to do the transformation;
\mintinline[breaklines=true, tabsize=2, breaksymbolleft=]{text}{CALL(fun, args) -> ESEQ(MOVE(TEMP t, CALL(fun, args)), TEMP t)}

\hypertarget{why-eseq-nodes-are-an-issue}{%
\subsection{\texorpdfstring{Why
\mintinline[breaklines=true, tabsize=2, breaksymbolleft=]{text}{ESEQ}
nodes are an
issue?}{Why  nodes are an issue?}}\label{why-eseq-nodes-are-an-issue}}

Clearly in case of simple
\mintinline[breaklines=true, tabsize=2, breaksymbolleft=]{text}{ESEQ(s, e)},
statement
\mintinline[breaklines=true, tabsize=2, breaksymbolleft=]{text}{s} can
have direct or side effects on an expression
\mintinline[breaklines=true, tabsize=2, breaksymbolleft=]{text}{e}.

Remedy is as shown in figure 5.1 (basically lifting them higher and
higher until they become
\mintinline[breaklines=true, tabsize=2, breaksymbolleft=]{text}{SEQ}
nodes).

\begin{figure}
\centering
\iph{0.70}{assets/canon1.png}
\caption{ESEQ Removal. Image source: \cite{tigerbook}}
\end{figure}

The transformation is done in three stages: First, a tree is rewritten
into a list of canonical trees without
\mintinline[breaklines=true, tabsize=2, breaksymbolleft=]{text}{SEQ} or
\mintinline[breaklines=true, tabsize=2, breaksymbolleft=]{text}{ESEQ}
nodes; then this list is grouped into a set of basic blocks, which
contain no internal jumps or labels; then the basic blocks are ordered
into a set of traces in which every
\mintinline[breaklines=true, tabsize=2, breaksymbolleft=]{text}{CJUMP}
is immediately followed by its false label. This will become clear when
seeing the well documented code.

\section{Instruction Selection}

\tfbox{\ghref{assem.sml} \\ \ghref{codegen.sml} \\ \ghref{accessConv.sml}}

\hypertarget{instruction-representation}{%
\subsection{Instruction
Representation}\label{instruction-representation}}

Now that we have done Canonisation, our compiler will now call code
generator. It will basically convert the body into assembly language
code with the restriction that we will still be using temporaries
instead of actual machine registers. In this form, source registers will
be labelled
\mintinline[breaklines=true, tabsize=2, breaksymbolleft=]{text}{`si} and
destination registers as
\mintinline[breaklines=true, tabsize=2, breaksymbolleft=]{text}{`di}. So
our each instruction would be represented as an operation string having
these
\mintinline[breaklines=true, tabsize=2, breaksymbolleft=]{text}{`si},
\mintinline[breaklines=true, tabsize=2, breaksymbolleft=]{text}{`di}'s
where
\mintinline[breaklines=true, tabsize=2, breaksymbolleft=]{text}{`si} is
indexed from
\mintinline[breaklines=true, tabsize=2, breaksymbolleft=]{text}{src}
list, and
\mintinline[breaklines=true, tabsize=2, breaksymbolleft=]{text}{`di} are
indexed from
\mintinline[breaklines=true, tabsize=2, breaksymbolleft=]{text}{dst}
list. This is captured as
\mintinline[breaklines=true, tabsize=2, breaksymbolleft=]{text}{datatype instr}
in
\href{https://www.github.com/sourabh2311/btp/tree/master/Compiler/assem.sml}{\mintinline[breaklines=true, tabsize=2, breaksymbolleft=]{text}{assem.sml}}.

\begin{minted}[breaklines=true, tabsize=2, breaksymbolleft=]{sml}
datatype instr = 
            OPER of {assem: string, dst: temp list, src: temp list, jump: label list option}
    | LABEL of {assem: string, lab: Temp.label}
    | MOVE of {assem: string,  dst: temp, src: temp}
\end{minted}

An \mintinline[breaklines=true, tabsize=2, breaksymbolleft=]{text}{OPER}
holds an assembly-language instruction
\mintinline[breaklines=true, tabsize=2, breaksymbolleft=]{text}{assem},
a list of operand registers
\mintinline[breaklines=true, tabsize=2, breaksymbolleft=]{text}{src},
and a list of result registers
\mintinline[breaklines=true, tabsize=2, breaksymbolleft=]{text}{dst}.
Either of these lists may be empty. Operations that always fall through
to the next instruction have
\mintinline[breaklines=true, tabsize=2, breaksymbolleft=]{text}{jump = NONE};
other operations have a list of ``target'' labels to which they may jump
(this list must explicitly include the next instruction if it is
possible to fall through to it).

A \mintinline[breaklines=true, tabsize=2, breaksymbolleft=]{text}{LABEL}
is a point in a program to which jumps may go. It has an
\mintinline[breaklines=true, tabsize=2, breaksymbolleft=]{text}{assem}
component showing how the label will look in the assembly-language
program, and a
\mintinline[breaklines=true, tabsize=2, breaksymbolleft=]{text}{lab}
component identifying which label-symbol was used.

A \mintinline[breaklines=true, tabsize=2, breaksymbolleft=]{text}{MOVE}
is like an
\mintinline[breaklines=true, tabsize=2, breaksymbolleft=]{text}{OPER},
but must perform only data transfer. Then, if the
\mintinline[breaklines=true, tabsize=2, breaksymbolleft=]{text}{dst} and
\mintinline[breaklines=true, tabsize=2, breaksymbolleft=]{text}{src}
temporaries are assigned to the same register, the
\mintinline[breaklines=true, tabsize=2, breaksymbolleft=]{text}{MOVE}
can later be deleted (as done in my register allocator).

\hypertarget{code-generation}{%
\subsection{Code Generation}\label{code-generation}}

Finding the appropriate machine instructions to implement a given
intermediate representation tree is the job of the instruction selection
phase of a compiler.

We can express a machine instruction as a fragment of an IR tree, called
a tree pattern as denoted in the figure below. Then instruction
selection becomes the task of tiling the tree with a minimal set of tree
patterns.

\begin{figure}
\centering
\iph{0.7}{assets/is1.png}
\caption{Tree Patterns showing arithmetic and memory instructions. Image source: \cite{tigerbook}}
\end{figure}

The very first entry is not really an instruction, but expresses the
idea that a
\mintinline[breaklines=true, tabsize=2, breaksymbolleft=]{text}{TEMP}
node is implemented as a register, so it can ``produce a result in a
register'' without executing any instructions at all.

For each instruction, the tree-patterns it implements are shown. Some
instructions correspond to more than one tree pattern; the alternate
patterns are obtained for commutative operators (+ and *), and in some
cases where a register or constant can be zero
(\mintinline[breaklines=true, tabsize=2, breaksymbolleft=]{text}{LOAD}
and
\mintinline[breaklines=true, tabsize=2, breaksymbolleft=]{text}{STORE}).
Here we abbreviate the tree diagrams slightly:
\mintinline[breaklines=true, tabsize=2, breaksymbolleft=]{text}{BINOP(PLUS, x, y)}
nodes will be written as
\mintinline[breaklines=true, tabsize=2, breaksymbolleft=]{text}{+(x, y)},
and the actual values of
\mintinline[breaklines=true, tabsize=2, breaksymbolleft=]{text}{CONST}
and
\mintinline[breaklines=true, tabsize=2, breaksymbolleft=]{text}{TEMP}
nodes will not always be shown.

Consider the instruction
\mintinline[breaklines=true, tabsize=2, breaksymbolleft=]{text}{a[i] := x}.
Two possible tilings for this instruction is shown in the diagram.
(Remember that
\mintinline[breaklines=true, tabsize=2, breaksymbolleft=]{text}{a} is
really the frame offset of the pointer to an array)

\begin{figure}
\centering
\iph{0.7}{assets/is2.png}
\caption{A tree tiled in two ways. Image source: \cite{tigerbook}}
\end{figure}

\hypertarget{maximal-munch-algorithm}{%
\subsection{Maximal Munch Algorithm}\label{maximal-munch-algorithm}}

Suppose we could give each kind of instruction a cost. Then we could
define an optimum tiling as the one whose tiles sum to the lowest
possible value. An optimal tiling is one where no two adjacent tiles can
be combined into a single tile of lower cost. If there is some tree
pattern that can be split into several tiles of lower combined cost,
then we should remove that pattern from our catalog of tiles before we
begin. Every optimum tiling is also optimal, but not vice versa.

The algorithm for optimal tiling is called \textbf{Maximal Munch}. It is
quite simple. Starting at the root of the tree, find the largest tile
that fits. Cover the root node - and perhaps several other nodes near
the root - with this tile, leaving several subtrees. Now repeat the same
algorithm for each subtree. As each tile is placed, the corresponding
instruction is generated. The Maximal Munch algorithm generates the
instructions in reverse order - after all, the instruction at the root
is the first to be generated, but it can only execute after the other
instructions have produced operand values in registers. The ``largest
tile'' is the one with the most nodes. If two tiles of equal size match
at the root, then the choice between them is arbitrary. Maximal Munch is
quite straightforward to implement in ML. Simply write two recursive
functions,
\mintinline[breaklines=true, tabsize=2, breaksymbolleft=]{text}{munchStm}
for statements and
\mintinline[breaklines=true, tabsize=2, breaksymbolleft=]{text}{munchExp}
for expressions. Each clause of
\mintinline[breaklines=true, tabsize=2, breaksymbolleft=]{text}{munchExp}
will match one tile. The clauses are ordered in order of tile preference
(biggest tiles first); ML's pattern-matching always chooses the first
rule that matches.

This algorithm is implemented in the file
\href{https://www.github.com/sourabh2311/btp/tree/master/Compiler/codegen.sml}{\mintinline[breaklines=true, tabsize=2, breaksymbolleft=]{text}{codegen.sml}}

\section{Liveness Analysis and Interference Graph}

\tfbox{\ghref{flowgraph.sml} \\ \ghref{makegraph.sml} \\ \ghref{liveness.sml}}

\hypertarget{liveness-analysis}{%
\subsection{Liveness Analysis}\label{liveness-analysis}}

Two temporaries
\mintinline[breaklines=true, tabsize=2, breaksymbolleft=]{text}{a} and
\mintinline[breaklines=true, tabsize=2, breaksymbolleft=]{text}{b} can
fit into the same register, if
\mintinline[breaklines=true, tabsize=2, breaksymbolleft=]{text}{a} and
\mintinline[breaklines=true, tabsize=2, breaksymbolleft=]{text}{b} are
never ``in use'' at the same time.

We say a variable is live if it holds a value that may be needed in the
future, so this analysis is called liveness analysis. To perform
analyses on a program, it is often useful to make a control-flow graph.
Each statement in the program is a node in the flow graph; if statement
\mintinline[breaklines=true, tabsize=2, breaksymbolleft=]{text}{x} can
be followed by statement
\mintinline[breaklines=true, tabsize=2, breaksymbolleft=]{text}{y} there
is an edge from
\mintinline[breaklines=true, tabsize=2, breaksymbolleft=]{text}{x} to
\mintinline[breaklines=true, tabsize=2, breaksymbolleft=]{text}{y}.

In this compiler, flow graph is as represented in
\href{https://www.github.com/sourabh2311/btp/tree/master/Compiler/flowgraph.sml}{\mintinline[breaklines=true, tabsize=2, breaksymbolleft=]{text}{flowgraph.sml}}.

Similarly
\href{https://www.github.com/sourabh2311/btp/tree/master/Compiler/flowgraph.sml}{\mintinline[breaklines=true, tabsize=2, breaksymbolleft=]{text}{makegraph.sml}}
constructs this graph.

A variable is live on an edge if there is a directed path from that edge
to a use of the variable that does not go through any
\mintinline[breaklines=true, tabsize=2, breaksymbolleft=]{text}{def}. A
variable is live-in at a node if it is live on any of the in-edges of
that node; it is live-out at a node if it is live on any of the
out-edges of the node.

Liveness of node is calculated as shown$^{\cite{tigerbook}}$:

\(in[n] = use[n] \cup (out[n] \setminus def[n])\)

\(out[n] = \cup_{s \in succ[n]} in[s]\)

A condition that prevents
\mintinline[breaklines=true, tabsize=2, breaksymbolleft=]{text}{a} and
\mintinline[breaklines=true, tabsize=2, breaksymbolleft=]{text}{b} being
allocated to the same register is called an interference.

The most common kind of interference is caused by overlapping live
ranges when
\mintinline[breaklines=true, tabsize=2, breaksymbolleft=]{text}{a} and
\mintinline[breaklines=true, tabsize=2, breaksymbolleft=]{text}{b} are
both live at the same program point, then they cannot be put in the same
register.

\hypertarget{interference-graph}{%
\subsection{Interference Graph}\label{interference-graph}}

Interference graph is created with vertices as temporaries and edges as
follows$^{\cite{tigerbook}}$:-

\begin{enumerate}
\def\labelenumi{\arabic{enumi}.}
\item
  At any nonmove instruction that defines a variable \(a\), where the
  live-out variables are \(b_1, \dots, b_j\), add interference edges
  \((a, b_1), \dots, (a, b_j)\).
\item
  At a move instruction \(a \leftarrow c\), where variables
  \(b_1, \dots, b_j\) are live-out, add interference edges
  \((a, b_1), \dots, (a, b_j)\) for any \(b_i\) that is not the same as
  \(c\).
\end{enumerate}

\begin{center}\rule{0.5\linewidth}{\linethickness}\end{center}

All this is handled in
\href{https://www.github.com/sourabh2311/btp/tree/master/Compiler/liveness.sml}{\mintinline[breaklines=true, tabsize=2, breaksymbolleft=]{text}{liveness.sml}}

\section{Register Allocation}

\tfbox{\ghref{regalloc.sml} \\ \ghref{regalloc.sig}}

Now that we have made our interference graph, where each edge denotes
that its endpoints must go in different registers, thus, our problem
reduced into coloring our graph with
\mintinline[breaklines=true, tabsize=2, breaksymbolleft=]{text}{K} (\#
no. of available registers) colors such that no two end points have same
colour.

As of now this is implemented by below mentioned algorithm \textbf{but}
later I would implement the exact algorithm which is given in book.
Algorithm is implemented in the file
\href{https://www.github.com/sourabh2311/btp/tree/master/Compiler/regalloc.sml}{\mintinline[breaklines=true, tabsize=2, breaksymbolleft=]{text}{regalloc.sml}}
and
\href{https://www.github.com/sourabh2311/btp/tree/master/Compiler/regalloc.sig}{\mintinline[breaklines=true, tabsize=2, breaksymbolleft=]{text}{regalloc.sig}}

Note that graph coloring problem is not fixed parameter tractable with
respect to number of colors.

\hypertarget{algorithm}{%
\subsection{Algorithm}\label{algorithm}}

Recursive algorithm is defined as below, assume that we are given a set
of instructions each of which may contain temporaries which might have
been already mapped to some register.

\begin{enumerate}
\def\labelenumi{\arabic{enumi}.}
\item
  Construct interference graph given the list of instructions.
\item
  Start with vertices (temporaries) which have not been assigned a
  colour (register).
\item
  Such vertices whose \# Neighbours \(<\)
  \mintinline[breaklines=true, tabsize=2, breaksymbolleft=]{text}{K} can
  be removed from the graph as we would always have a color available
  for them. Thus this divides our list of vertices into two viz.,
  (\mintinline[breaklines=true, tabsize=2, breaksymbolleft=]{text}{simplify},
  \mintinline[breaklines=true, tabsize=2, breaksymbolleft=]{text}{potentialSpill}).
\item
  Now we should process these nodes in
  \mintinline[breaklines=true, tabsize=2, breaksymbolleft=]{text}{simplify},
  let
  ``\mintinline[breaklines=true, tabsize=2, breaksymbolleft=]{text}{stack}''
  denote the set of simplified nodes.
\item
  Repeat until
  \mintinline[breaklines=true, tabsize=2, breaksymbolleft=]{text}{potentialSpill}
  nodes become empty

  \begin{enumerate}
  \def\labelenumii{\arabic{enumii}.}
  \item
    We ``simplify'' each node in our
    \mintinline[breaklines=true, tabsize=2, breaksymbolleft=]{text}{simplify}
    list by removing it from
    \mintinline[breaklines=true, tabsize=2, breaksymbolleft=]{text}{simplify}
    list and adding it to
    \mintinline[breaklines=true, tabsize=2, breaksymbolleft=]{text}{stack}
    and decrementing the degree of its neighbors which are not in
    \mintinline[breaklines=true, tabsize=2, breaksymbolleft=]{text}{stack}
    (\mintinline[breaklines=true, tabsize=2, breaksymbolleft=]{text}{stack}
    nodes are already removed) and also which aren't already colored.
    Note that when we decrement the degree of the neighbors, if degree
    becomes less then
    \mintinline[breaklines=true, tabsize=2, breaksymbolleft=]{text}{K}
    that means we can add this node to
    \mintinline[breaklines=true, tabsize=2, breaksymbolleft=]{text}{simplify}
    and remove it from
    \mintinline[breaklines=true, tabsize=2, breaksymbolleft=]{text}{potentialSpill}.
  \item
    If
    \mintinline[breaklines=true, tabsize=2, breaksymbolleft=]{text}{potentialSpill}
    is empty then done, o/w select a node to spill, choose that which
    has maximum neighbors and should have accessed temps/memory least
    number of times. Assign it to
    \mintinline[breaklines=true, tabsize=2, breaksymbolleft=]{text}{simplify}
    and remove it from
    \mintinline[breaklines=true, tabsize=2, breaksymbolleft=]{text}{potentialSpill}.
  \end{enumerate}
\item
  Now process
  \mintinline[breaklines=true, tabsize=2, breaksymbolleft=]{text}{stack}
  from top to bottom. For an element, colors available to it are total
  minus those taken by its neighbors, if colors are available then
  assign else we add it to actualSpill.
\item
  If by now actualSpill is empty then we stop else we rewrite the
  program and repeat.
\end{enumerate}