\chapter{Introduction to RISC V and Runtime}

\section{Choosing a Simulator}

In case of MIPS, it was straight forward to compile and run the assembly code but in case of RISC V, I had to struggle for some time as I couldn't find an appropriate documentation for it. After many iterations, I found the most suitable simulator, namely, \href{https://github.com/TheThirdOne/rars}{RARS}.

Please see the corresponding description of system calls \href{https://github.com/TheThirdOne/rars/wiki/Environment-Calls}{here}. It as well have a nice companion \href{https://github.com/TheThirdOne/rars/wiki}{documentation} which I expect one to read before proceeding further with the report.

There are other ways to compile and run RISC V code:

\begin{itemize}

  \item \href{http://www.kvakil.me/venus/}{Venus}, \href{https://github.com/kvakil/venus}{Github repo}. Note that for system calls, their argument register is different, see \href{https://github.com/TheThirdOne/rars/issues/45}{this}. 

  \item \texttt{spike} (sort of an official simulator), installed when using \texttt{riscv-gnu-toolchain}. Note it was as well required to install \href{https://github.com/riscv/riscv-pk}{pk}. System calls are different than RARS, basically it follows linux system calls. Can see these system calls \href{https://github.com/riscv/riscv-pk/blob/master/pk/syscall.c}{here} and \href{https://github.com/riscv/riscv-pk/blob/master/pk/syscall.h}{here}. And linux system calls \href{http://man7.org/linux/man-pages/man2/syscalls.2.html}{here}, note that system calls of interest can be concisely seen \href{https://rv8.io/syscalls.html}{here}.
  
  So to compile and run the program, do: (Don't know if this is the intended way but after a lot of trial and error, I found this)
  
  \begin{minted}{bash}
  >> riscv64-unknown-elf-as -o filename.o filename.s 
  >> riscv64-unknown-elf-ld -o filename filename.o 
  >> spike pk filename
  \end{minted}

\end{itemize}

\section{Basic Examples}

Few things to keep in mind:

\begin{itemize}
  \item Note that dont use \texttt{\$reg}, instead simply use \texttt{reg}.
  \item Always have two sections, one for data and another for text.
  \item During \texttt{ecall} all registers besides the output are guaranteed not to change.
  \item Put return value in \texttt{a0}.
  \item When we want our register value to be saved across system call, we can simply put it in stack instead of using registers $s_i$ as anyway we would have to save their value first in stack so no change in overhead.
  \item \texttt{a7} is used to tell which system call.
  \item Save \texttt{ra} register if executing \texttt{jal} inside a function.
\end{itemize}

\subsection{Hello World}

\begin{minted}{asm}
.data # Tell the assembler we are defining data not code

msg: # Label this position in memory so it can be referred to in our code
  .string "hello world" # Copy the string "hello world" into memory 

.text # Tell the assembler that we are writing code (text) now 

start:
  li a7, 4 # li means to Load Immediate and we want to load the value 4 into register a7
  la a0, msg  # la is similar to li, but works for loading addresses
  ecall
  li a7, 10  # Exit call
  ecall
\end{minted}

To get the same code working using spike. 

\begin{minted}{asm}
.globl _start # We must need to give _start, .globl helps to see it outside this file
.data 
str:   
  .string "Hello World!\n" 

.text 
_start: 

  li a0, 1   
  la a1, str 
  li a2, 13  # length of the string as required for linux system call. We can write a function which will determine the length of the string by checking for terminating null character.
  li a7, 64  
  ecall 

  li a0, 0   # The exit code we will be returning is 0
  li a7, 93  # Again we need to indicate what system call we are making and this time we are calling exit(93)
  ecall 
\end{minted}

\subsection{Saving callee save registers}
\begin{minted}{asm}
.data

  bef: .string "Before modification, value is: "
  dur: .string "\nInside function, value is: "
  aft: .string "\nAfter function call, value is: "

.text 

main:
  addi s0, zero, 1
  # Print bef
  la a0, bef
  li a7, 4
  ecall
  # Print int
  li a7, 1
  mv a0, s0
  ecall
  jal increment
  # Print aft
  la a0, aft
  li a7, 4
  ecall
  # Print int
  li a7, 1
  mv a0, s0
  ecall
  # Exit
  li a7, 10
  ecall



increment:
  addi sp, sp, -4
  sw s0, 0(sp) # '0' denotes the offset, in case of 0, we can simply omit it.
  addi s0, s0, 1
  # Print string
  la a0, dur
  li a7, 4
  ecall
  # Print the incremented integer
  mv a0, s0
  li a7, 1
  ecall
  lw s0, 0(sp)
  addi sp, sp, 4
  jr ra
\end{minted}

\section{Runtime}

\tfbox{\href{\git runtime.s}{runtime.s}}

There are some standard functions which our Tiger program can use (like \texttt{print}, etc.). They are written in \texttt{runtime.s}. I could have as well used \href{https://www.cs.princeton.edu/~appel/modern/spim/runtime.s}{file} provided by author but as I want to augment it further and implement things my way, I decided to write runtime myself. Each of the functions of runtime is explained in the code for easy understanding. 
