\documentclass{beamer}

\usepackage{xcolor}  
\usepackage{environ}
\usepackage{minted}
\usepackage{tikzsymbols}

% map defaulf block to oldblock
\let\oldblock\block
\let\endoldblock\endblock

% change block by adding smallskip
\renewenvironment{block}[1]
    {\begin{oldblock}{#1}
        \smallskip
    }
    { 
    \end{oldblock}
    }

\usetheme[progressbar=frametitle]{metropolis}

\makeatletter
\setlength{\metropolis@titleseparator@linewidth}{2pt}
\setlength{\metropolis@progressonsectionpage@linewidth}{2pt}
\setlength{\metropolis@progressinheadfoot@linewidth}{2pt}
\makeatother

\newcommand{\iph}[2]{
    \includegraphics[width=#1\textwidth,height=#1\textheight,keepaspectratio]{#2}
}
\newcommand{\ph}[1]{
    \includegraphics[width=0.5\textwidth,height=0.5\textheight,keepaspectratio]{#1}
}

\title{Tiger to RISC V Compiler}

\author{Sourabh Aggarwal\\ Mentor: Dr. Piyush P. Kurur}

\institute[IIT Palakkad] % (optional, but mostly needed)
{
  Department of Computer Science And Engineering\\
  IIT Palakkad
}

\date{September 2019}

\subject{BTP Interim Presentation}

\setminted{breaklines=true, tabsize=2, breaksymbolleft=, fontsize=\footnotesize}

\begin{document}

\begin{frame}
  \titlepage
\end{frame}

\begin{frame}[fragile]
  \frametitle{Tiger}
  \begin{minted}{sml}
function try (c : int) = 
( 
  if c = N
  then printboard ()
  else 
    for r := 0 to N - 1 do 
      if row[r] = 0 & diag1[r + c] = 0 & diag2[r + 7 - c] = 0 then 
      (
        row[r] := 1; diag1[r + c] := 1; diag2[r + 7 - c] := 1;
        col[c] := r;
        try(c + 1);
        row[r] := 0; diag1[r + c] := 0; diag2[r + 7 - c] := 0
      )
  )
  \end{minted}
 % \iph{1.1}{../Task1/myPR.png}
\end{frame}

\begin{frame}[fragile]
  \frametitle{RISC V}
  \begin{minted}{asm}
# str1 < str2 ?
stringLess:
    stringLessLoop:
        lb a2 (a0)
        lb a3 (a1)
        blt a2, a3  stringLessA
        bgt a2, a3  stringLessB
        # If we have reached this point that means both are equal and if one of them is zero that means other is aswell 0, so in case strings are equal, I must return 0.
        beqz a2, stringLessB
        addi a0, a0, 1
        addi a1, a1, 1
        j stringLessLoop
    stringLessA:
        li a0, 1
        jr ra
    stringLessB: li a0, 0 jr ra 
  \end{minted}
\end{frame}

\begin{frame}[fragile]
  \frametitle{Modifying And Extending Upon Previous Compiler}
  \begin{itemize}
    \item Wrote compiler to compile Tiger to MIPS last semester (didn't expect to get it working). 
    \pause
    \item BUT evidently there were issues.
  \end{itemize}
\end{frame}

\begin{frame}[fragile]
  \frametitle{Issues / Challenges}
  \begin{itemize}
  \item The number of arguments that can be given to a function is exactly the number of available argument registers. This is to be fixed to support any number of function arguments.
  \pause
  \item Register allocation currently works only on saved registers and temporaries, this can be augmented to use free argument registers as well. (Subjective move)
  \pause
  \item Basic blocks have to optimized with optimizations like constant propagation, constant folding, etc.
  \pause
  \item String comparison has to be made as simple as "str1 $>$ str2", etc.,  instead of calling the string comparison functions to determine it.
  \pause
  \item To implement Garbage collection.
  \end{itemize}
\end{frame}
\begin{frame}
  \frametitle{Issues / Challenges}
  \begin{itemize}
  \item To implement ability to include pre-written code (header) files.
  \pause
  \item To implement additional arithmetic operations like left/right shift.
  \pause
  \item Register allocation algorithm has to be implemented with better heuristics.
  \pause
  \item Error messages has to improved in semantic phase. 
  \pause
  \item Add support for compile time (initial) arguments.
  \pause
  \item And much more is possible, as like in a commercial product.
  \end{itemize}
\end{frame}
\begin{frame}
  \frametitle{Goal for this Semester}
  This semester, I am mainly focusing on:-
  \begin{itemize}
    \pause
    \item Understanding RISC V.
    \pause 
    \item Recollecting all that was done last semester (3 month gap \Xey). % Reason of using tikzsymbols
    \pause
    \item Refactoring, fixing bugs and translating everything to RISC V.
    \pause
    \item Writing good documentation of the code for easy recollection.
    \pause 
    \item Travis script for automated testing.
    \pause 
    \item And of course, if there is time, I'll work on the previous mentioned issues else they are to be focused on next semester.
  \end{itemize}
\end{frame}

\begin{frame}[fragile]
  \frametitle{Compiler Course: Phase I - Building AST}
  Example:-
  \begin{minted}{sml}
| LET decs IN exps END              (
  A.LetExp({decs = decs, body = A.SeqExp(exps), pos = LETleft})
  )
  \end{minted}
  Included many files, viz. 
  \begin{itemize}
    \item parse.sml 
    \item errormsg.sml 
    \item tiger.lex 
    \item table.sml, table.sig 
    \item symbol.sml 
    \item tiger.grm, absyn.sml 
  \end{itemize}
\end{frame}

\begin{frame}[fragile]
  \frametitle{Compiler Course: Phase II - Building Intermeditate Representation Tree}
  \begin{minted}{sml}
datatype stm = SEQ of stm * stm 
            | LABEL of label 
            | JUMP of exp * label list 
            | CJUMP of relop * exp * exp * label * label 
            | MOVE of exp * exp 
            | EXP of exp 
and exp = BINOP of binop * exp * exp 
            | MEM of exp 
            | TEMP of Temp.temp 
            | ESEQ of stm * exp 
            | NAME of label 
            | CONST of int 
            | CALL of exp * exp list 
  \end{minted}
  
\end{frame}
\begin{frame}[fragile]
  \frametitle{Compiler Course: Phase II - Building Intermeditate Representation Tree}
  Included many files, viz. 
  \begin{itemize}
    \item types.sml 
    \item env.sml, env.sig 
    \item temp.sml, temp.sig 
    \item findescape.sml 
    \item tree.sml 
    \item risc.sml 
    \item translate.sml 
    \item semant.sml 
  \end{itemize}
\end{frame}

\begin{frame}[fragile]
  \frametitle{Till Now}
  \begin{itemize}
    \item Understood RISC V and rewrote "runtime" in RISC V. 
    \pause
    \item Worked out till Phase II of Compiler.
  \end{itemize}
\end{frame}
\begin{frame}[standout]
Thanks!
\end{frame}   
\end{document}
